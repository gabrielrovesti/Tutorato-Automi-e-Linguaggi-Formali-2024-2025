\documentclass[12pt,a4paper]{article}

% Encoding e Font
\usepackage[T1]{fontenc}
\usepackage[utf8]{inputenc}
\usepackage{lmodern}

% Pacchetti per la matematica
\usepackage{amsmath,amssymb,amsthm}

% Margini e layout
\usepackage{geometry}
\geometry{top=2.5cm, bottom=2.5cm, left=2.5cm, right=2.5cm}

% Link cliccabili e riferimenti ipertestuali
\usepackage{hyperref}
\hypersetup{
    colorlinks = true,
    linkcolor  = blue,
    citecolor  = blue,
    urlcolor   = blue
}

% Pacchetto per elenchi personalizzati
\usepackage{enumitem}

% TikZ per disegnare automi e diagrammi
\usepackage{tikz}
\usetikzlibrary{automata,arrows,positioning,calc,decorations.pathreplacing}

% Pacchetto per colorare il testo
\usepackage{xcolor}

% Scatole colorate per i suggerimenti e le note
\usepackage{tcolorbox}
\tcbuselibrary{skins}

% Ambiente per suggerimenti
\newtcolorbox{suggerimento}{
  colback=blue!10,
  colframe=blue!50!black,
  title=\textbf{Suggerimento},
  fonttitle=\bfseries
}

% Ambiente per errori comuni
\newtcolorbox{errorecomune}{
  colback=red!10,
  colframe=red!50!black,
  title=\textbf{Errore comune},
  fonttitle=\bfseries
}

% Ambiente per concetti chiave
\newtcolorbox{concettochiave}{
  colback=green!10,
  colframe=green!50!black,
  title=\textbf{Concetto chiave},
  fonttitle=\bfseries
}

% Ambiente personalizzato per risoluzioni
\newtcolorbox{risoluzione}{
  colback=gray!10,
  colframe=gray!50!black,
  title=\textbf{Procedimento di risoluzione},
  fonttitle=\bfseries
}

% Comandi personalizzati
\newcommand{\N}{\mathbb{N}}
\newcommand{\Z}{\mathbb{Z}}
\newcommand{\A}{\Sigma}
\newcommand{\eps}{\varepsilon}
\newcommand{\states}{\mathcal{Q}}
\newcommand{\startstate}{q_0}
\newcommand{\finalstates}{\mathcal{F}}
\newcommand{\powerset}{\mathcal{P}}
\newcommand{\eclose}[1]{\text{ECLOSE}(#1)}
\newcommand{\lang}[1]{\mathcal{L}(#1)}

%%%%%%%%%%%%%%%%%%%%%%%%%%%%%%%%%%%%%%%%%%%%%%%%%%%%%%%%%%%%%%%%%%%%%%%%%%%%%%%
% FRONTESPIZIO
%%%%%%%%%%%%%%%%%%%%%%%%%%%%%%%%%%%%%%%%%%%%%%%%%%%%%%%%%%%%%%%%%%%%%%%%%%%%%%%
\title{
  \vspace{1cm}
  \Huge \textbf{Pumping Lemma e Linguaggi non Regolari}\\[0.5cm]
  \Large \textbf{Tecniche, Strategie e Casi Studio}\\
  \normalsize Tutorato 4: Scelta strategica della parola e applicazione del Pumping Lemma
}

\author{\textbf{Gabriel Rovesti} \\
Corso di Laurea in Informatica - Università degli Studi di Padova
}

\date{Anno Accademico 2024-2025}

%%%%%%%%%%%%%%%%%%%%%%%%%%%%%%%%%%%%%%%%%%%%%%%%%%%%%%%%%%%%%%%%%%%%%%%%%%%%%%%
% INIZIO DOCUMENTO
%%%%%%%%%%%%%%%%%%%%%%%%%%%%%%%%%%%%%%%%%%%%%%%%%%%%%%%%%%%%%%%%%%%%%%%%%%%%%%%
\begin{document}

\maketitle
\tableofcontents
\newpage

\section{Introduzione al Pumping Lemma}

Il Pumping Lemma è uno strumento fondamentale nella teoria degli automi e dei linguaggi formali, utilizzato principalmente per dimostrare che un determinato linguaggio \textbf{non} è regolare.

\begin{concettochiave}
Mentre esistono molti metodi per dimostrare che un linguaggio è regolare (costruire un DFA, un NFA, o un'espressione regolare), è spesso più difficile dimostrare che un linguaggio non è regolare. Il Pumping Lemma fornisce un metodo sistematico per questa dimostrazione.
\end{concettochiave}

\subsection{Intuizione alla base del Pumping Lemma}

L'idea fondamentale dietro il Pumping Lemma deriva da una limitazione intrinseca degli automi a stati finiti: la loro memoria finita.

\begin{suggerimento}
Intuizione: Se un automa a stati finiti legge una stringa sufficientemente lunga, deve necessariamente ripetere alcuni stati. Questa ripetizione crea un "ciclo" che può essere ripetuto un numero arbitrario di volte, generando nuove stringhe che l'automa accetta.
\end{suggerimento}

Questa proprietà di "pompabilità" è una caratteristica necessaria di tutti i linguaggi regolari, e la sua assenza dimostra che un linguaggio non può essere regolare.

\section{Enunciato Formale del Pumping Lemma}

\begin{concettochiave}
\textbf{Pumping Lemma:} Se $L$ è un linguaggio regolare, allora esiste una costante $p > 0$ (pumping length) tale che ogni stringa $s \in L$ con $|s| \geq p$ può essere scomposta in $s = xyz$ dove:
\begin{enumerate}
    \item $|y| > 0$ (la parte "pompabile" non è vuota)
    \item $|xy| \leq p$ (i primi due pezzi insieme hanno lunghezza al massimo $p$)
    \item $\forall i \geq 0$, $xy^iz \in L$ (la stringa ottenuta ripetendo $y$ un numero arbitrario di volte appartiene ancora a $L$)
\end{enumerate}
\end{concettochiave}

\subsection{Dimostrazione del Lemma}

\begin{risoluzione}
\begin{enumerate}
    \item Se $L$ è regolare, esiste un DFA $A$ con un certo numero $n$ di stati che lo riconosce
    \item Sia $p = n$ (la pumping length è uguale al numero di stati)
    \item Consideriamo una stringa $w \in L$ con lunghezza $|w| \geq p$
    \item Durante il riconoscimento di $w$, l'automa attraversa $|w|+1$ stati (incluso quello iniziale)
    \item Poiché $|w| \geq p = n$, almeno $n+1$ stati vengono visitati
    \item Per il principio della piccionaia, l'automa deve ripetere almeno uno stato
    \item Siano $q_i$ e $q_j$ la prima ripetizione di uno stato con $i < j \leq p$
    \item Definiamo $x = w[1 \ldots i]$, $y = w[i+1 \ldots j]$, $z = w[j+1 \ldots |w|]$
    \item Allora $|y| > 0$ (perché $i < j$) e $|xy| \leq p$ (perché $j \leq p$)
    \item Poiché lo stato $q_i = q_j$ è ripetuto, il "ciclo" $y$ può essere ripetuto qualsiasi numero di volte, e l'automa raggiungerà comunque uno stato finale dopo aver letto $z$
    \item Quindi, per ogni $i \geq 0$, $xy^iz \in L$
\end{enumerate}
\end{risoluzione}

\section{Il Gioco del Pumping Lemma}

Un modo efficace per applicare il Pumping Lemma è vederlo come un "gioco" tra due giocatori: uno che sostiene che il linguaggio è regolare (Avversario) e uno che sostiene che non lo è (Dimostratore).

\begin{concettochiave}
\textbf{Gioco del Pumping Lemma:}
\begin{enumerate}
    \item \textbf{Avversario} sceglie una costante $p > 0$ (pumping length)
    \item \textbf{Dimostratore} seleziona una stringa $w \in L$ con $|w| \geq p$
    \item \textbf{Avversario} decompone $w = xyz$ con $|y| > 0$ e $|xy| \leq p$
    \item \textbf{Dimostratore} trova un valore $i \geq 0$ tale che $xy^iz \not\in L$
    \item Se il Dimostratore riesce a trovare tale $i$, allora ha dimostrato che $L$ non è regolare
\end{enumerate}
\end{concettochiave}

\begin{suggerimento}
Nella struttura del gioco, il Dimostratore deve essere in grado di trovare un contro-esempio per \textbf{qualsiasi} decomposizione proposta dall'Avversario, non solo per alcune di esse.
\end{suggerimento}

\section{Scelta Strategica della Parola}

La scelta della stringa $w$ è cruciale per l'applicazione efficace del Pumping Lemma. La stringa deve essere:
\begin{itemize}
    \item Abbastanza lunga ($|w| \geq p$)
    \item Strutturata in modo tale che qualsiasi decomposizione porti a una contraddizione
    \item Semplice da analizzare
\end{itemize}

\subsection{Criteri per la scelta ottimale}

\begin{risoluzione}
Per scegliere la stringa $w$ ottimale:
\begin{enumerate}
    \item Identifica la proprietà fondamentale che caratterizza il linguaggio
    \item Costruisci una stringa che soddisfa questa proprietà in modo "minimale"
    \item Assicurati che pompando qualsiasi parte all'inizio della stringa si violi la proprietà fondamentale
    \item La stringa deve costringere l'Avversario a decomporre in un modo prevedibile
    \item Deve essere facilmente parametrizzabile rispetto a $p$
\end{enumerate}
\end{risoluzione}

\begin{concettochiave}
Le stringhe più efficaci spesso hanno una struttura che forza l'Avversario a decomporre in un punto specifico, dove qualsiasi pompaggio altera una proprietà fondamentale del linguaggio.
\end{concettochiave}

\subsection{Tecniche di scelta per linguaggi comuni}

\begin{suggerimento}
\textbf{Strategie per classi di linguaggi specifiche:}
\begin{itemize}
    \item Per linguaggi del tipo $\{a^n b^n\}$: scegli $w = a^p b^p$
    \item Per linguaggi con condizioni numeriche: scegli $w$ al "limite" della condizione
    \item Per linguaggi con strutture bilanciate: scegli $w$ perfettamente bilanciata
    \item Per linguaggi con parole palindrome: scegli $w$ con una struttura asimmetrica facilmente alterabile
    \item Per linguaggi con condizioni di divisibilità: scegli $w$ che soddisfa esattamente la condizione
\end{itemize}
\end{suggerimento}

\section{Casi Studio: Esempi di Applicazione}

\subsection{Caso 1: $L = \{a^n b^n \mid n \geq 0\}$}

\begin{risoluzione}
\textbf{Dimostrazione che $L$ non è regolare:}
\begin{enumerate}
    \item Assumiamo per assurdo che $L$ sia regolare
    \item Per il Pumping Lemma, esiste $p > 0$ tale che ogni stringa in $L$ di lunghezza $\geq p$ può essere pompata
    \item Scegliamo $w = a^p b^p \in L$ (ha lunghezza $2p \geq p$)
    \item Per qualsiasi decomposizione $w = xyz$ con $|y| > 0$ e $|xy| \leq p$:
    \begin{itemize}
        \item Poiché $|xy| \leq p$, sia $x$ che $y$ devono essere contenuti nei primi $p$ caratteri di $w$
        \item Quindi $y$ contiene solo caratteri $a$ (nessun $b$)
        \item Sia $y = a^k$ con $k > 0$
    \end{itemize}
    \item Scegliamo $i = 2$, ottenendo $xy^2z = a^{p+k}b^p$
    \item Questa stringa ha $p+k$ caratteri $a$ ma solo $p$ caratteri $b$
    \item Quindi $xy^2z \not\in L$ (non soddisfa la condizione $n_a = n_b$)
    \item Contraddizione con l'ipotesi che $L$ sia regolare
\end{enumerate}
\end{risoluzione}

\begin{suggerimento}
In questo esempio, la scelta di $w = a^p b^p$ è ottimale perché forza $y$ a contenere solo $a$. Si può anche usare $i = 0$ per ottenere meno $a$ che $b$, con lo stesso risultato.
\end{suggerimento}

\subsection{Caso 2: $L = \{a^i b^j \mid i > j \geq 0\}$}

\begin{risoluzione}
\textbf{Dimostrazione che $L$ non è regolare:}
\begin{enumerate}
    \item Assumiamo per assurdo che $L$ sia regolare con pumping length $p$
    \item Scegliamo $w = a^{p+1} b^p \in L$ (ha $i = p+1 > j = p$)
    \item Per qualsiasi decomposizione $w = xyz$ con $|y| > 0$ e $|xy| \leq p$:
    \begin{itemize}
        \item Poiché $|xy| \leq p$ e ci sono $p+1$ caratteri $a$ all'inizio, $xy$ cade interamente nella sezione di $a$
        \item Quindi $y = a^k$ per qualche $k > 0$
    \end{itemize}
    \item Scegliamo $i = 0$, ottenendo $xy^0z = xz = a^{p+1-k}b^p$
    \item Se $k = 1$, otteniamo $a^p b^p$ con lo stesso numero di $a$ e $b$
    \item Se $k > 1$, otteniamo una stringa con meno $a$ che $b$
    \item In entrambi i casi, $xy^0z \not\in L$ perché non soddisfa la condizione $i > j$
    \item Contraddizione con l'ipotesi che $L$ sia regolare
\end{enumerate}
\end{risoluzione}

\begin{concettochiave}
Notare come in questo caso la scelta strategica di $w = a^{p+1}b^p$ rappresenti il "caso limite" della disuguaglianza $i > j$. Rimuovendo anche solo un carattere $a$ (con $i = 0$), la condizione viene immediatamente violata.
\end{concettochiave}

\subsection{Caso 3: $L = \{a^{n^2} \mid n \geq 0\}$}

\begin{risoluzione}
\textbf{Dimostrazione che $L$ non è regolare:}
\begin{enumerate}
    \item Assumiamo per assurdo che $L$ sia regolare con pumping length $p$
    \item Scegliamo $w = a^{p^2} \in L$ (lunghezza $p^2 \geq p$ per $p \geq 1$)
    \item Per qualsiasi decomposizione $w = xyz$ con $|y| > 0$ e $|xy| \leq p$:
    \begin{itemize}
        \item Sia $y = a^k$ per qualche $k > 0$ (la stringa contiene solo $a$)
    \end{itemize}
    \item Scegliamo $i = 2$, ottenendo $xy^2z = a^{p^2+k}$
    \item Dobbiamo dimostrare che $p^2+k$ non è un quadrato perfetto per $1 \leq k \leq p$:
    \begin{itemize}
        \item $p^2 < p^2+k < p^2+2p+1 = (p+1)^2$ per $1 \leq k \leq p$
        \item Quindi $p^2+k$ è strettamente compreso tra due quadrati consecutivi
        \item Quindi $p^2+k$ non è un quadrato perfetto
    \end{itemize}
    \item Quindi $xy^2z \not\in L$, contraddizione
\end{enumerate}
\end{risoluzione}

\begin{suggerimento}
Per linguaggi che coinvolgono funzioni matematiche ($n^2$, $2^n$, numeri primi, ecc.), è essenziale sfruttare le proprietà matematiche di queste funzioni nella dimostrazione.
\end{suggerimento}

\subsection{Caso 4: $L = \{ww \mid w \in \{a,b\}^*\}$}

\begin{risoluzione}
\textbf{Dimostrazione che $L$ non è regolare:}
\begin{enumerate}
    \item Assumiamo per assurdo che $L$ sia regolare con pumping length $p$
    \item Scegliamo $w = a^pb^pa^pb^p \in L$ (è la duplicazione di $a^pb^p$)
    \item Per qualsiasi decomposizione $w = xyz$ con $|y| > 0$ e $|xy| \leq p$:
    \begin{itemize}
        \item Poiché $|xy| \leq p$, $xy$ è contenuto completamente nella prima sezione $a^p$
        \item Quindi $y = a^k$ per qualche $k > 0$
    \end{itemize}
    \item Scegliamo $i = 2$, ottenendo $xy^2z = a^{p+k}b^pa^pb^p$
    \item Questa stringa non può essere scritta come $uu$ per alcun $u$:
    \begin{itemize}
        \item Se fosse $uu$, la lunghezza di $u$ sarebbe $(2p+k+2p)/2 = 2p+k/2$
        \item Se $k$ è dispari, questo non è un intero
        \item Se $k$ è pari, avremmo $u = a^{p+k/2}b^pa^{k/2}$ e la seconda metà sarebbe $a^{p-k/2}b^p$, che sono diverse
    \end{itemize}
    \item Quindi $xy^2z \not\in L$, contraddizione
\end{enumerate}
\end{risoluzione}

\section{Casi Studio Avanzati}

\subsection{Caso 5: Linguaggio di sezioni distinte}

\begin{risoluzione}
Consideriamo il linguaggio $Y = \{w\mid w = x_1\#x_2\# \cdots \#x_k \text{ per } k \geq 0, \text{ dove ogni } x_i \in 1^*, \text{ e } x_i \neq x_j \text{ per } i \neq j\}$.

\textbf{Dimostrazione che $Y$ non è regolare:}
\begin{enumerate}
    \item Assumiamo per assurdo che $Y$ sia regolare con pumping length $p$
    \item Scegliamo $w = 1^1\#1^2\#1^3\#\ldots\#1^p\#1^{p+1} \in Y$
    \item Questa stringa ha $p+1$ sezioni tutte diverse tra loro
    \item Per qualsiasi decomposizione $w = xyz$ con $|y| > 0$ e $|xy| \leq p$:
    \begin{itemize}
        \item Poiché $|xy| \leq p$, $xy$ cade all'interno delle prime sezioni della stringa
    \end{itemize}
    \item Consideriamo i casi possibili:
    \begin{itemize}
        \item Se $y$ contiene solo caratteri $1$ all'interno di una sezione, con $i = 2$ aumentiamo la lunghezza di quella sezione, creando una duplicazione con un'altra sezione
        \item Se $y$ contiene il simbolo $\#$ con alcuni $1$ adiacenti, con $i = 0$ fondiamo due sezioni, creando una sezione che non era presente originariamente
        \item Se $y$ contiene più simboli $\#$, con $i = 2$ alteriamo completamente la struttura delle sezioni
    \end{itemize}
    \item In tutti i casi, $xy^iz \not\in Y$ per qualche $i$, contraddizione
\end{enumerate}
\end{risoluzione}

\begin{concettochiave}
Questo linguaggio richiede una "memoria illimitata" per verificare che tutte le sezioni siano diverse tra loro, proprietà che un automa a stati finiti non può avere.
\end{concettochiave}

\subsection{Caso 6: Linguaggio delle somme binarie}

\begin{risoluzione}
Consideriamo $ADD = \{x=y+z\mid x,y,z \text{ sono numeri binari, e } x \text{ è la somma di } y \text{ e } z\}$.

\textbf{Dimostrazione che $ADD$ non è regolare:}
\begin{enumerate}
    \item Assumiamo per assurdo che $ADD$ sia regolare con pumping length $p$
    \item Scegliamo $w = 10^p=10^p+0 \in ADD$ (rappresenta l'equazione $2^p = 2^p + 0$)
    \item Per qualsiasi decomposizione $w = xyz$ con $|y| > 0$ e $|xy| \leq p$:
    \begin{itemize}
        \item Poiché $|xy| \leq p$ e la prima parte della stringa è $10^p=$, $xy$ cade nella prima parte
    \end{itemize}
    \item Consideriamo i casi:
    \begin{itemize}
        \item Se $y$ contiene solo $0$ nella parte $10^p$, con $i = 2$ otteniamo $10^{p+k}=10^p+0$ che rappresenta $2^{p+k} = 2^p + 0$, falsa perché $2^{p+k} > 2^p$
        \item Se $y$ contiene il $1$ iniziale, con $i = 0$ otteniamo $0^{p-j}=10^p+0$ che è falsa
        \item Se $y$ contiene il simbolo $=$, con $i = 0$ eliminiamo il simbolo $=$, producendo una stringa malformata
    \end{itemize}
    \item In tutti i casi, esiste un $i$ tale che $xy^iz \not\in ADD$, contraddizione
\end{enumerate}
\end{risoluzione}

\begin{suggerimento}
Per linguaggi che coinvolgono operazioni aritmetiche, la strategia migliore è scegliere stringhe che rappresentano equazioni semplici ma che diventano false quando vengono pompate.
\end{suggerimento}

\section{Errori Comuni nell'Applicazione del Pumping Lemma}

\begin{errorecomune}
\textbf{Errore 1: Interpretazione errata del quantificatore universale}

È l'Avversario (che difende la regolarità) a scegliere la decomposizione $w = xyz$. Il Dimostratore deve dimostrare che \emph{per ogni possibile decomposizione} esiste un $i$ tale che $xy^iz \not\in L$.

Molti studenti erroneamente pensano di poter scegliere una decomposizione specifica, ma questo non è corretto.
\end{errorecomune}

\begin{errorecomune}
\textbf{Errore 2: Confondere la direzione logica}

Il Pumping Lemma fornisce una condizione \emph{necessaria} ma non sufficiente per i linguaggi regolari:
\begin{itemize}
    \item Se un linguaggio è regolare, allora soddisfa il Pumping Lemma
    \item Se un linguaggio non soddisfa il Pumping Lemma, allora non è regolare
\end{itemize}

Attenzione: non è vero che se un linguaggio soddisfa il Pumping Lemma, allora è regolare!
\end{errorecomune}

\begin{errorecomune}
\textbf{Errore 3: Scelta non ottimale della stringa $w$}

Scegliere una stringa troppo complicata o che non forzi l'Avversario a decomporre in un punto specifico può rendere la dimostrazione molto più difficile o impossibile.
\end{errorecomune}

\section{Casi Particolari e Tecniche Avanzate}

\subsection{Minimum Pumping Length}

Il concetto di "minimum pumping length" (la più piccola pumping length possibile per un linguaggio regolare) è utile per analizzare la complessità di un linguaggio.

\begin{risoluzione}
\textbf{Determinazione della Minimum Pumping Length:}
\begin{enumerate}
    \item Per un linguaggio regolare $L$, la minimum pumping length $p$ è il più piccolo intero tale che ogni stringa $w \in L$ con $|w| \geq p$ può essere pompata
    \item Per trovarla:
    \begin{itemize}
        \item Costruire il DFA minimo per $L$
        \item Analizzare le stringhe più corte che non possono essere pompate
        \item Trovare la lunghezza della stringa più lunga tra queste
    \end{itemize}
\end{enumerate}

\textbf{Esempi:}
\begin{itemize}
    \item $000\,1^*$: minimum pumping length = 4
    \item $0^*\,1^*$: minimum pumping length = 1
    \item $\{1011\}$ (un singolo elemento): minimum pumping length = 5
    \item $(01)^*$: minimum pumping length = 3
\end{itemize}
\end{risoluzione}

\subsection{Tecniche per linguaggi con condizioni complesse}

\begin{suggerimento}
\textbf{Per linguaggi con condizioni condizionali} (del tipo "se ... allora ..."):
\begin{itemize}
    \item Scegliere una stringa che soddisfi esattamente la condizione "se"
    \item Assicurarsi che pompando si alteri la condizione "allora"
    \item Esempio: Per $\{a^i b^j c^k \mid i,j,k \geq 0 \text{ e se } i=1 \text{ allora } j=k\}$, scegliere $w = ab^pc^p$
\end{itemize}
\end{suggerimento}

\begin{suggerimento}
\textbf{Per linguaggi con conteggio di pattern}:
\begin{itemize}
    \item Costruire una stringa con un numero bilanciato di pattern
    \item Assicurarsi che pompando si alteri questo equilibrio
    \item Esempio: Per $\{w \in \{a,b\}^* \mid w \text{ contiene lo stesso numero di } ab \text{ e } ba\}$, scegliere $w = (ab)^p(ba)^p$
\end{itemize}
\end{suggerimento}

\section{Riepilogo e Consigli Pratici}

\begin{concettochiave}
\textbf{Passi fondamentali per dimostrare la non regolarità con il Pumping Lemma:}
\begin{enumerate}
    \item Assumere per assurdo che il linguaggio $L$ sia regolare
    \item Considerare la pumping length $p$ garantita dal Pumping Lemma
    \item Scegliere strategicamente una stringa $w \in L$ con $|w| \geq p$
    \item Analizzare tutte le possibili decomposizioni $w = xyz$ con $|y| > 0$ e $|xy| \leq p$
    \item Trovare un valore $i$ tale che $xy^iz \not\in L$ per ogni decomposizione
    \item Concludere che $L$ non è regolare
\end{enumerate}
\end{concettochiave}

\subsection{Checklist operativa}

\begin{risoluzione}
\textbf{Checklist per una dimostrazione efficace:}
\begin{itemize}
    \item La stringa $w$ è abbastanza lunga? ($|w| \geq p$)
    \item La stringa $w$ appartiene al linguaggio? ($w \in L$)
    \item Hai analizzato \emph{tutte} le possibili decomposizioni?
    \item Hai verificato che $|y| > 0$ e $|xy| \leq p$ per tutte le decomposizioni considerate?
    \item Hai trovato un valore $i$ tale che $xy^iz \not\in L$ per ogni decomposizione?
    \item Hai dimostrato formalmente che $xy^iz \not\in L$?
\end{itemize}
\end{risoluzione}

\subsection{Consigli finali}

\begin{suggerimento}
\begin{itemize}
    \item Inizia con una chiara comprensione della proprietà fondamentale che caratterizza il linguaggio
    \item Scegli la stringa più semplice possibile che soddisfi i requisiti
    \item Spesso $i = 0$ (rimozione) o $i = 2$ (duplicazione) sono sufficienti
    \item Organizza la dimostrazione per casi se necessario
    \item Verifica sempre che la stringa pompata violi effettivamente la definizione del linguaggio
\end{itemize}
\end{suggerimento}

\section{Conclusioni}

Il Pumping Lemma è uno strumento potente nella teoria dei linguaggi formali, che ci permette di dimostrare rigorosamente che certi linguaggi non possono essere riconosciuti da automi a stati finiti.

La sua applicazione efficace richiede:
\begin{itemize}
    \item Una comprensione profonda della struttura del linguaggio in esame
    \item Una scelta strategica della stringa da pompare
    \item Un'analisi sistematica di tutte le possibili decomposizioni
    \item Una dimostrazione formale che il pompaggio porta a stringhe fuori dal linguaggio
\end{itemize}

\begin{concettochiave}
Padroneggiare il Pumping Lemma non solo aiuta a classificare i linguaggi, ma sviluppa anche importanti capacità di ragionamento formale e di dimostrazione matematica che sono fondamentali in informatica teorica.
\end{concettochiave}

\end{document}